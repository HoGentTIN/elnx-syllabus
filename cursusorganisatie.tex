\chapter{Cursusorganisatie}
\label{cursusorganisatie}

\section{Praktische afspraken}
\label{praktische-afspraken}

Wat betreft praktische afspraken, regelingen, verwachtingen, enz. in
verband met deze cursus zijn dit de enige geldige bronnen van
informatie:

\begin{itemize}
\item
  De
  \href{https://bamaflexweb.hogent.be/BMFUIDetailxOLOD.aspx?a=68521\&b=5\&c=1}{ECTS-fiche}
  van het opleidingsonderdeel
\item
  Deze studiewijzer
\item
  Aankondigingen en documenten op Chamilo
\end{itemize}

Indien er ergens twijfel over bestaat, of er is iets niet duidelijk,
neem dan zo snel mogelijk contact op met de vaktitularis
(\href{mailto:bert.vanvreckem@hogent.be}{Bert Van Vreckem}), hetzij
tijdens de contactmomenten, hetzij via e-mail. Indien nuttig of nodig
wordt het antwoord als aankondiging op Chamilo doorgegeven aan alle
studenten. De ervaring leert dat de onderlinge communicatie tussen
studenten via Facebook leidt tot verwarring, foute informatie en
discussie achteraf. Gebruik dus aub de officiële kanalen zodat we tot
een open en correcte communicatie rond deze cursus kunnen komen.

Jullie zijn zelf verantwoordelijk voor het opvolgen en lezen van alle
aankondigingen.

\section{Opzet lessen}
\label{opzet-lessen}

In deze cursus zijn er \textbf{geen theorielessen}. Bij aanvang van het
semester krijg je een uitleg over het opzet van de cursus en over
praktische afspraken, maar in principe is dat de enige keer dat er een
plenaire uitleg komt. Er worden tal van (vrij beschikbare) studiematerialen aangereikt die je in staat moeten stellen de nodige achtergrondinformatie zelf te verwerken.

De lectoren van de leerlijn systeem- en netwerkbeheer zijn de
mening toegedaan dat ons vakdomein er één is dat je enkel kan eigen
maken door \textbf{praktijkervaring} op te doen. Bovendien evolueert het
vakgebied zo snel dat bijleren en informatie over nieuwe
technologieën of tools opzoeken en verwerken een vast onderdeel is van de job.

Tijdens de contactmomenten willen we daarom maximaal tijd vrij maken om
met de praktijk bezig te zijn aan de hand van labo-taken. Dat vereist
een zekere mate van zelfstandigheid, maar dat is precies ook een
attitude die verwacht wordt van een systeembeheerder. Je neemt dus zelf
initiatief om de nodige kennis te vergaren en zoekt naar oplossingen
voor de problemen die je ongetwijfeld zal tegenkomen. Help elkaar
daarin: samenwerken en kennis delen wordt van harte aangemoedigd. De
lector is uiteraard ook beschikbaar om je bij te staan als je toch vast
komt te zitten en kan je tips geven of verwijzen naar geschikte
studiematerialen.

\section{Evaluatie}
\label{evaluatie}

De evaluatie van dit vak is volledig op basis van niet-periodegebonden
evaluatie. Je wordt beoordeeld op de volgende opdrachten:

\begin{itemize}
\item
  Labo-opdracht opzetten infrastructuur met een configuration management
  system
\item
  Taak rond actualiteit in het vakgebied
\item
  Troubleshooting van netwerkservices
\item
  Rapportering en documentatie
\end{itemize}

Verderop wordt in meer detail uitgelegd wat precies verwacht wordt. De
toekenning van het examencijfer gebeurt op basis van ``rubrics''~\autocite{Andrade2000}. In een
tabel worden een aantal criteria opgesomd, waar je aan moet voldoen. Je
kan ``voldoen'' op verschillende niveaus, meer bepaald ``bekwaam'',
``gevorderd'', ``deskundig'', of in het slechtste geval ``nog niet
bekwaam''. Voor elk niveau wordt beschreven wat dat precies inhoudt. Het
volledige evaluatieschema wordt tijdig op Chamilo gepubliceerd.

Om te slagen voor dit vak moet je aantonen dat je voor \textbf{alle} technische
criteria minstens ``bekwaam'' bent. Met andere woorden, wie voor één
criterium ``nog niet bekwaam'' bevonden wordt, kan niet slagen. De
niet-technische criteria (bv. rapportering) wegen minder zwaar mee, maar
als je hier ``nog niet bekwaam'' bevonden wordt, zal dit het
examencijfer negatief beïnvloeden.

Je bekwaamheid aantonen gebeurt aan de hand van volgende deliverables:

\begin{itemize}
\item
  De broncode, door elke student bijgehouden in een toegewezen Git
  repository
\item
  Gedetailleerde labo-verslagen, eveneens bijgehouden in Git
\item
  Demonstratie van deelresultaten aan de lector tijdens het semester
\item
  Demonstratie van het eindresultaat aan de lector tijdens de
  examenperiode
\end{itemize}

\textbf{Merk op} dat je zowel een werkend product moet opleveren (= broncode) als de verslagen indienen én demo's geven. Als één van de deliverables ontbreekt, wordt de opdracht beschouwd als niet gerealiseerd. Je moet minstens elke drie weken deelresulaten opleveren en persoonlijk demonstreren aan de lector (voor studenten TILE: zie verder).

Wat betreft het respecteren van deadlines is art. 34 van het FOER van kracht. Niet tijdig deelresultaten demonstreren wordt in termen van dat artikel beschouwd als het niet respecteren van tussentijdse deadlines en wordt als dusdanig gesanctioneerd. Eén van de opdrachten (zoals hierboven opgesomd) niet realiseren leidt tot een afwezigheid voor die opdracht en bijgevolg ook voor het gehele opleidingsonderdeel, conform het examenreglement.

\subsection{Afstandsleren (TILE)}
\label{afstandsleren-tile}

Wie de cursus volgt via afstandsleren voert dezelfde opdrachten uit en
wordt op dezelfde manier geëvalueerd. Dat betekent ook dat je regelmatig
deelresultaten demonstreert aan de lector. Dat kan op verschillende
manieren:

\begin{itemize}
\item aan de hand van een screencast (bv. publiceren op Youtube, link doorsturen naar de lector),
\item via videoconferencing (Google Hangouts of Skype, na afspraak),
\item tijdens de contactmomenten voor TILE-studenten, als de lector daar aanwezig is (en enkel na afspraak),
\item tijdens de normale contactmomenten, zowel op campus Aalst (dinsdag 9:15 - 12:30, lokaal GAARB.0.029) als Gent (woensdag 9:15 - 12:30, lokaal B4.037).
\end{itemize}

Voor de troubleshooting-oefeningen (zie verder) kan je met de lector een moment afspreken waarop je tijd hebt om hier aan te werken. Op het afgesproken tijdstip krijg je een downloadlink naar de opgave en dien je voor de afgesproken deadline (in principe 3 uur na ontvangst van de opgave) in op Chamilo.

Vragen kan je stellen per e-mail of op de contactmomenten voor TILE.

\subsection{Tweede examenkans}
\label{tweede-examenkans}

Wie niet slaagt dient voor de tweede zittijd een individuele opdracht uit te voeren. De precieze opdracht hangt af van je evaluatie. Meer bepaald zal je moet kunnen aantonen dat je voor alle technische criteria minstens ``bekwaam'' bent. Deze opdracht wordt vastgelegd op de feedback.

\textbf{Wie vóór het ingaan van het intersemestrieel reces geen afspraken maakt ivm een individuele opdracht, geeft daarmee te kennen niet te willen deelnemen aan de tweede zittijd.}

\section{Gebruikte hard- en software}
\label{gebruikte-hard--en-software}

De aard van deze cursus maakt dat het essentieel is dat je beschikt over voldoende krachtige hardware. Op campus Schoonmeersen kan je gebruik maken van de klaspc's, maar blijft het belangrijk om ook thuis of op kot te kunnen verder werken. Enkele aanbevelingen:

\begin{itemize}
\item Processor: Intel i7 of equivalent (een i5 is voldoende, maar wie een i3-processor heeft wordt sterk aangeraden te investeren)
\item Geheugen: 8 GiB RAM
\item Harde schijf: 40 GB vrije schijfruimte is ruim voldoende
\end{itemize}

Voor het uitwerken van de labo-taken maak je gebruik van de hieronder opgesomde software (in principe de laatste stabiele versies bij aanvang van het semester). Als je sommige van de opgesomde applicaties al geïnstalleerd hebt, werk die dan bij tot de laatste versie.

\begin{itemize}
\item Een \textbf{degelijke} teksteditor: Sublime, Notepad++, Vim, enz.
  \begin{itemize}
  \item Let op! Stel je editor zo in dat tekst inspringt met \textbf{spaties} i.p.v. tabs en zet tabbreedte op 2 karakters.
  \end{itemize}
\item VirtualBox 5.x\footnote{\url{https://www.virtualbox.org/wiki/Downloads}} (installeer ook het ``Extension Pack'').
\item Vagrant\footnote{\url{http://vagrantup.com/}} \textgreater{}= 1.8.4
\item Git\footnote{\url{https://git-scm.com/downloads}}
\item Bash shell. Onder Windows, kan je de Bash shell die meegeleverd wordt met Git gebruiken. Mac gebruikers wordt aangeraden om een recente versie van Bash te installeren, bijvoorbeeld via HomeBrew.
\item Ansible\footnote{\url{https://docs.ansible.com/ansible/intro_installation.html}} \textgreater{}= 2.1 (enkel voor wie werkt met een Mac of Linux-laptop, Windows-gebruikers kunnen dit niet installeren maar voor hen is een workaround voorzien)
\item Optioneel:

  \begin{itemize}
    \item Een teksteditor voor Markdown\footnote{\url{http://daringfireball.net/projects/markdown/}}, bijvoorbeeld MarkdownPad\footnote{\url{http://markdownpad.com/}}, Texts\footnote{\url{http://www.texts.io/}}, enz.
  \item Pandoc\footnote{\url{http://johnmacfarlane.net/pandoc/}} en \LaTeX\ voor het
    omzetten van documentatie in Markdown naar PDF.
  \end{itemize}
\end{itemize}

Verder komt het volgende nog van pas:

\begin{itemize}
\item Een USB-stick of externe USB-schijf met voldoende ruimte (voor het bijhouden van VMs en broncode)
\item Een netwerkkabel en RJ-45 koppelstuk (netwerklokaal c.Schoonmeersen)
\item Oortjes voor het beluisteren/bekijken van videolessen, screencasts of videopresentaties zonder de rest te storen
\end{itemize}

\section{Wat moet je kennen en kunnen?}
\label{wat-moet-je-kennen-en-kunnen}

\subsection{Leerdoelen}
\label{leerdoelen}

De belangrijkste doelstellingen van deze cursus (die je ook vindt in de studiefiche\footnote{\url{https://bamaflexweb.hogent.be/BMFUIDetailxOLOD.aspx?a=77579&b=5&c=1}}) zijn:

\begin{itemize}
\item Netwerkservices kunnen opzetten in Linux, voor middelgrote tot grote ondernemingen (wat een bijzondere nadruk op automatisering impliceert);
\item Systematisch en grondig troubleshooten;
\item De meest relevante informatie kunnen opzoeken specifiek voor het systeem waar je mee werkt.
\end{itemize}

\subsection{Te bekijken videolessen, -tutorials, enz.}
\label{te-bekijken-videolessen--tutorials-enz.}

% TODO: is dit de beste plaats hiervoor?

\begin{itemize}
  \item SELinux for mere mortals~\autocite{Cameron2012}
  \item Ansible: Python-Powered Radically Simple IT Automation~\autocite{DeHaan2014}
  \item Vijf trends in systeembeheer, inleidingsles Enterprise Linux~\autocite{VanVreckem2013}
\end{itemize}

\subsection{Documentatie gebruiken}
\label{documentatie-gebruiken}

Je moet in staat zijn de juiste documentatie op te zoeken en te gebruiken, die relevant is voor het systeem waar je mee werkt, i.h.b. RHEL/CentOS 7.

\begin{itemize}
  \item RHEL/CentOS7 handleidingen, i.h.b. de System Administrator's Guide~\autocite{SvistunovEtAl2016}, Networking Guide~\autocite{JahodaEtAl2016} en SELinux User's and Administrator's Guide~\autocite{JahodaEtAl2016a}.
\item Ansible documentatie~\autocite{Ansible2016}, in het bijzonder

  \begin{itemize}
  \item Ansible Best Practices: \url{http://docs.ansible.com/playbooks_best_practices.html}
  \item Ansible Module Index: \url{http://docs.ansible.com/modules_by_category.html}
  \end{itemize}
\item Vagrant documentatie~\autocite{Hashicorp}.
\end{itemize}

\subsection{Systeembeheer RHEL/CentOS 7}
\label{systeembeheer-rhelcentos-7}

Je moet de juiste commando's kennen om RHEL/Centos \textbf{7} hosts te
beheren en te troubleshooten.

\begin{itemize}
\item Basiskennis uit Besturingssystemen (gebruikers, groepen, bestandspermissies, directorystructuur, package management, enz.)
\item Services beheren (\texttt{systemctl})
\item Firewalld beheren (\texttt{firewalld-cmd})
\item Secure Shell gebruiken (\texttt{ssh}, \texttt{scp}), inloggen ahv een SSH-sleutel
\item SELinux begrijpen en toepassen (Ančincová, 2014), meer bepaald:

  \begin{itemize}
  \item Status opvragen en wijzigen (\texttt{getenforce}, \texttt{setenforce})
  \item De verschillende modi begrijpen (\texttt{enforcing}, \texttt{permissive}, \texttt{disabled} )
  \item SELinux context van bestanden opvragen (\texttt{ls\ -Z}) en aanpassen (\texttt{chcon})
  \item Logs kunnen bekijken (\texttt{/var/log/audit/audit.log})
  \item Booleans kunnen opvragen en wijzigen (\texttt{getsebool}, \texttt{setsebool})
  \end{itemize}
\end{itemize}

Te kennen uit de RedHat manuals:

\begin{itemize}
\item System Administration Guide~\autocite{SvistunovEtAl2016}

  \begin{itemize}
  \item Basic system configuration (hst 1-4)
  \item Package Management (hst 5)
  \item Infrastructure Services (hst 6-7)
  \item Opzetten van netwerkservices: web (hst 9), mail (hst 10) , file/print (hst 12; \emph{NIET} 12.3)
  \item Monitoring and automation (hst 16, 18, 19)
  \end{itemize}

\item Networking Guide~\autocite{JahodaEtAl2016}

  \begin{itemize}
  \item Inleiding (hst 1, i.h.b. secties 1.3-4, 1.7-8-9)
  \item Command line (sectie 2.4)
  \item Hostnamen (hst 3)
  \item DHCP servers (hst 10)
  \item BIND DNS (hst 11)
  \end{itemize}
\item
  SELinux User's and Administrator's Guide~\autocite{JahodaEtAl2016a}

  \begin{itemize}
  \item Inleiding (hst 1)
  \item SELinux contexts (hst 2)
  \item Working with SELinux (hst 4, secties 1-6, 9)
  \item Troubleshooting (hst 10)
  \end{itemize}
\end{itemize}

\subsection{Vagrant}
\label{vagrant}

Je moet in staat zijn om Vagrant te gebruiken om nieuwe VMs op te starten en die te configureren.

\begin{itemize}
\item De inhoud een \texttt{Vagrantfile} kunnen interpreteren en aanpassen
\item Base boxes beheren (commando \texttt{vagrant\ box})
\item VMs beheren (\texttt{vagrant\ status}, \texttt{vagrant\ up}, \texttt{vagrant\ halt}, \texttt{vagrant\ reload}, \texttt{vagrant\ destroy}, enz.)
\item Inloggen op een Vagrant VM (\texttt{vagrant\ ssh})
\end{itemize}

\subsection{Ansible}
\label{ansible}

\begin{itemize}
\item Je moet in staat zijn bestaande Ansible-playbooks en rollen te interpreteren en aan te passen voor je eigen doeleinden
\item Je moet in staat zijn Ansible-rollen te schrijven voor RHEL/CentOS 7, meer bepaald

  \begin{itemize}
  \item De directorystructuur voor een rol aanmaken
  \item Een \emph{playbook} schrijven voor de uit te voeren taken (\texttt{tasks/main.yml}) en daarbij gebruik maken van Ansible modules
  \item Bestanden van het hostsysteem naar de VM kopiëren
  \item Templates definiëren en gebruiken
  \item Handlers definiëren
  \item Variabelen declareren en toekennen
  \end{itemize}

\item Verder moet je die rollen kunnen toekennen aan hosts:

  \begin{itemize}
    \item Een directorystructuur volgens de Ansible Best Practices\footnote{\url{http://docs.ansible.com/playbooks_best_practices.html}} aanhouden
  \item \texttt{site.yml} aanmaken
  \item De inventory-file aanmaken en groepen kunnen definiëren
  \item Variabelen voor individuele hosts (\texttt{host\_vars}) en groepen (\texttt{group\_vars}) initialiseren
  \end{itemize}
\end{itemize}
