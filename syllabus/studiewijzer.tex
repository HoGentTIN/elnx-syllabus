\chapter{Studiewijzer}%
\label{ch:studiewijzer}

De inhoud van deze studiewijzer geldt zowel voor de \textbf{reguliere} studenten als de studenten \textbf{afstandsleren (TILE)}. Waar nodig wordt duidelijk onderscheid gemaakt tussen zaken die enkel voor één van deze twee groepen gelden.

Wat betreft praktische afspraken, regelingen, verwachtingen, enz.\ in
verband met deze cursus zijn dit de enige geldige bronnen van
informatie:

\begin{itemize}
  \item De studiefichefiche van dit opleidingsonderdeel, te bekijken via de Chamilo-cursus of \url{https://hogent.be/studiefiches/};
  \item Deze studiewijzer;
  \item Documenten op Chamilo;
  \item Aankondigingen op Chamilo---deze worden ook telkens per e-mail naar de betrokken studenten gestuurd
\end{itemize}

Jullie zijn zelf verantwoordelijk voor het opvolgen en lezen van alle aankondigingen. Studenten worden geacht hun opleidingsgerelateerde e-mails regelmatig op te volgen.

Indien er ergens twijfel over bestaat, of er is iets niet duidelijk, neem dan zo snel mogelijk contact op met je lector. De aangewezen manieren worden opgesomd in Sectie~\ref{sec:studiebegeleiding-en-planning}.

De ervaring leert dat de onderlinge communicatie tussen studenten via Facebook leidt tot verwarring, foute informatie en overbodige discussie. Gebruik dus a.u.b.~de officiële kanalen zodat we tot een open en correcte communicatie rond deze cursus kunnen komen.

\section{Doel en plaats van de cursus in het curriculum}%
\label{sec:doel-en-plaats}

De bedoeling van deze cursus is om je in staat te stellen om op een betrouwbare, reproduceerbare en schaalbare manier netwerkdiensten in productie te brengen op Linux aan de hand van een Configuration Management System.

Wanneer je als systeembeheerder een serverpark van tientallen, honderden of zelfs duizenden machines (hetzij fysiek, hetzij virtueel) moet beheren, dan is het manueel opzetten, of zelfs scripten van de configuratie niet meer voldoende. In dit soort omgevingen wordt steevast gebruik gemaakt van Configuration Management Systems om de taaklast beheersbaar te houden. In deze cursus maken we gebruik van Ansible, omdat dit een voor beginners toegankelijk systeem is dat qua filosofie en logica aansluit bij klassieke shell scripts. In het werkveld is de kans groot dat je met andere Configuration Management Systems in aanraking komt, bijvoorbeeld Puppet of Chef.

Kennis over Linux wordt meer en meer gevraagd op de arbeidsmarkt en wie een goed resultaat haalt voor deze cursus kan overal aan de slag als Linux system engineer.

Binnen het curriculum volgt deze cursus op Besturingssystemen (en meer bepaald het onderdeel Linux). De vakinhoud van deze cursus wordt ook toegepast binnen Project 3: Systeembeheer.

\section{Leerdoelen en competenties}%
\label{sec:leerdoelen}

De belangrijkste doelstellingen van deze cursus (die je ook vindt in de studiefiche) zijn:

\begin{enumerate}
  \item Kan het opzetten van netwerkdiensten automatiseren met een configuratiebeheersysteem (configuration management system).
  \item Kan configuratieproblemen bij netwerkdiensten opsporen aan de hand van een systematische en grondige methodologie.
  \item Kan de meest geschikte documentatie voor een specifieke situatie (meer bepaald de Linux-distributie, softwareversies, enz.) opzoeken en gebruiken.
\end{enumerate}

Om deze doelstellingen te behalen, dien je de hieronder opgesomde competenties te verwerven.

\subsection{Configuration Management}%
\label{ssec:config-mgmt}

\begin{itemize}
  \item Je moet in staat zijn om Vagrant te gebruiken om nieuwe VMs op te starten en die te configureren.
  
  \begin{itemize}
    \item De inhoud een \texttt{Vagrantfile} kunnen interpreteren en aanpassen
    \item Base boxes beheren (commando \texttt{vagrant\ box})
    \item VMs beheren (\texttt{vagrant\ status}, \texttt{vagrant\ up}, \texttt{vagrant\ halt}, \texttt{vagrant\ reload}, \texttt{vagrant\ destroy}, enz.)
    \item Inloggen op een Vagrant VM (\texttt{vagrant\ ssh})
  \end{itemize}

  \item Je moet in staat zijn bestaande Ansible-playbooks en rollen te interpreteren en aan te passen voor je eigen doeleinden
  
  \item Verder moet je die rollen kunnen toekennen aan hosts:
  
  \begin{itemize}
    \item Een directorystructuur volgens de Ansible Best Practices\footnote{\url{http://docs.ansible.com/playbooks_best_practices.html}} aanhouden
    \item \texttt{site.yml} aanmaken
    \item De inventory-file aanmaken en groepen kunnen definiëren
    \item Variabelen voor individuele hosts (\texttt{host\_vars}) en groepen (\texttt{group\_vars}) initialiseren
  \end{itemize}
\end{itemize}

\subsection{Troubleshooting}%
\label{ssec:troubleshooting}

Je moet de juiste commando's kennen om RHEL/Centos \textbf{8} hosts te
beheren en te troubleshooten.

\begin{itemize}
  \item Basiskennis uit Besturingssystemen (gebruikers, groepen, bestandspermissies, directorystructuur, package management, enz.)
  \item Services beheren met \texttt{systemctl}
  \item Firewalld beheren met \texttt{firewalld-cmd}
  \item Secure Shell gebruiken (\texttt{ssh}, \texttt{scp}), inloggen ahv een SSH-sleutel
  \item SELinux begrijpen en toepassen, meer bepaald:
  
  \begin{itemize}
    \item Status opvragen en wijzigen (\texttt{getenforce}, \texttt{setenforce}, \texttt{/etc/selinux/config})
    \item De verschillende modi begrijpen (\texttt{enforcing}, \texttt{permissive}, \texttt{disabled})
    \item SELinux context van bestanden opvragen (\texttt{ls\ -Z}) en aanpassen (\texttt{chcon}, \texttt{restorecon})
    \item Logs kunnen bekijken en problemen opsporen (\texttt{/var/log/audit/audit.log})
    \item Booleans kunnen opvragen en wijzigen (\texttt{getsebool}, \texttt{setsebool})
  \end{itemize}
\end{itemize}

\subsection{Documentatie gebruiken}%
\label{ssec:documentatie-gebruiken}

Je gebruikt de  om relevante informatie op te zoeken over de taken die je moet uitvoeren. Je bent in het bijzonder vertrouwd met de inhoud van de handleidingen hieronder opgesomd, d.w.z.\ dat je in grote lijnen weet wat er in staat en dat je er snel in kan terugvinden wat je nodig hebt.

\begin{itemize}
  \item Handleidingen voor RHEL8\footnote{\url{https://access.redhat.com/documentation/en-US/Red_Hat_Enterprise_Linux/8/}} of CentOS 8:
  \begin{itemize}
    \item System Administration:
    \begin{itemize}
      \item Deploying Different Types of Servers
      \item Configuring and Managing Networking
    \end{itemize}
    \item Security:
    \begin{itemize}
      \item Using SELinux
    \end{itemize}
  \end{itemize}
  \item Ansible documentatie~\autocite{Ansible2016}, in het bijzonder
  
  \begin{itemize}
    \item Ansible Best Practices\footnote{\url{http://docs.ansible.com/playbooks_best_practices.html}}
    \item Ansible Module Index\footnote{\url{http://docs.ansible.com/modules_by_category.html}}
  \end{itemize}
  \item Vagrant documentatie~\autocite{Hashicorp}.
\end{itemize}

\section{Leerinhoud}%
\label{sec:leerinhoud}

De cursus bestaat uit drie onderdelen:

\begin{enumerate}
  \item Realiseren van een labo-opdracht die het opzetten van infrastructuur a.h.v. Ansible inhoudt (wordt verderop in deze studiewijzer de \emph{hoofdopdracht} genoemd). Je kan daarbij als student vrij kiezen uit drie concrete opdrachten (zie Hoofdstuk~\ref{ch:opdrachten}).
  \item Volgen van de actualiteit in het vakgebied en dit toepassen op je hoofdopdracht (wordt verder de \emph{actualiteitsopdracht} genoemd)
  \item Een aantal \emph{troubleshooting}-labo's.
\end{enumerate}

\section{Leermateriaal}%
\label{sec:leermateriaal}

\subsection{Handboeken, handleidingen, enz.}%
\label{ssec:handboeken}

Zoals eerder vermeld, gebruiken we als leermateriaal in de eerste plaats de handleidingen van de software die we gebruiken in de cursus. Al deze leermaterialen zijn gratis on-line raadpleegbaar.

\begin{itemize}
  \item RHEL 8 documentatie
  \item Ansible documentatie~\autocite{Ansible2016}
  \item Vagrant documentatie~\autocite{Hashicorp}
  \item DNS for Rocket Scientists~\autocite{Aitchison2015}
  \item Samba User Documentation\footnote{\url{https://wiki.samba.org/index.php/User_Documentation}}
  \item Analyzing and Solving Samba Problems~\autocite{CarterEtAl2010}
  \item The Samba Checklist~\autocite{TrigdellEtAl2010}
\end{itemize}

Verderop in deze syllabus vind je nog aanvullende informatie die je ook verder op weg kan helpen.

Als herhaling van je Linux-kennis uit Besturingssystemen kan je opnieuw het boek van~\textcite{Cobbaut2015} raadplegen.

Optioneel is het boek ``Ansible for Devops'' van~\textcite{Geerling2016} warm aanbevolen. Je koopt dit best als e-boek, want op die manier krijg je ook toegang krijgt tot nieuwe edities die regelmatig verschijnen (b.v.~bij de release van een nieuwe versie van Ansible).

\subsection{Video}%
\label{ssec:video}

Screencasts en presentaties:

\begin{itemize}
  \item Crash Course on Vagrant~\autocite{Weissig2014}
  \item 19 Minutes with Ansible~\autocite{Weissig2015}
  \item Ansible: Python-Powered Radically Simple IT Automation~\autocite{DeHaan2014}
  \item SELinux for mere mortals~\autocite{Cameron2012}
  \item Een fileserver opzetten met Samba~\autocite{VanVreckem2014}
  \item Linux Troubleshooting~\autocite{VanVreckem2015b}
\end{itemize}

\subsection{Hardware}%
\label{ssec:hardware}

De aard van deze cursus maakt dat het essentieel is dat je beschikt over voldoende krachtige hardware. Aanbevolen is minstens:

\begin{itemize}
  \item Processor: Intel i5 7e generatie of equivalent/beter
  \item Geheugen: 16 GiB RAM
  \item Harde schijf: 80 GB vrije schijfruimte
\end{itemize}

Verder komt het volgende nog van pas:

\begin{itemize}
  \item Een USB-stick of externe USB-schijf met voldoende ruimte (voor het bijhouden van VMs en broncode)
  \item Een Ethernet-kabel (netwerklabo GAARB.0.032 en GSCHB.4.037)
  \item Oortjes voor het beluisteren/bekijken van videolessen, screencasts of videopresentaties zonder de rest te storen
\end{itemize}

\subsection{Software}%
\label{ssec:software}

Voor het uitwerken van de labo-taken maak je gebruik van de hieronder opgesomde software, in principe de laatste stabiele versies bij aanvang van het semester. Als je sommige van de opgesomde applicaties al geïnstalleerd hebt, werk die dan bij tot de laatste versie.

\begin{itemize}
  \item Een \textbf{package manager}, zoals je op een Linux-systeem hebt, kan het installeren van de verschillende nodige softwarepakketen enorm vereenvoudigen. Voor Windows is er Chocolatey\footnote{\url{https://chocolatey.org/}}, voor MacOS X is er Homebrew\footnote{\url{https://brew.sh/}}. Instructies vind je verderop.
  \item Een \textbf{goede teksteditor} met syntaxkleuren: Visual Studio Code, Sublime Text, Notepad++, Atom, Vim, Brackets, enz. \textbf{Geen} Notepad, Wordpad of Word!
  \begin{itemize}
    \item Let op! Stel je editor zo in dat tekst inspringt met \textbf{spaties} i.p.v.\ tabs en zet tabbreedte in op 2 karakters.
    \item In Linux hoef je niet specifiek iets extra te installeren. De default teksteditor, Gedit, heeft syntax-colouring. Je kan eventueel ook werken met Vim, installeer dan de package \texttt{vim-enhanced}. Vim is niet eenvoudig om mee aan de slag te gaan, maar eens je de vele commando's leert kennen, wordt het een bijzonder krachtige en productieve werkomgeving! Visual Studio Code is ook beschikbaar voor Linux.
  \end{itemize}
  \item VirtualBox\footnote{\url{https://www.virtualbox.org/wiki/Downloads}} (installeer ook het ``Extension Pack'').
  \item Vagrant\footnote{\url{http://vagrantup.com/}}
  \item Git\footnote{\url{https://git-scm.com/downloads}} client (onder Windows, installeer ook Git Bash)
  \item Bash shell. Onder Windows, kan je de Bash shell die meegeleverd wordt met Git gebruiken. Mac-gebruikers wordt aangeraden om een recente versie van Bash te installeren via HomeBrew.
  \item Ansible\footnote{\url{https://docs.ansible.com/ansible/intro_installation.html}} (enkel voor wie werkt met een Mac of Linux-laptop. Wie Windows gebruikt kan dit niet installeren. Voor hen is een workaround voorzien)
\end{itemize}

Installatie op Windows met Chocolatey:

\begin{minted}{console}
C:\> choco install git
C:\> choco install vscode
C:\> choco install virtualbox
C:\> choco install vagrant
\end{minted}

Installatie op MacOS X met Homebrew:

\begin{minted}{console}
$ brew install bash
$ brew install git
$ brew cask install visual-studio-code
$ brew cask install virtualbox
$ brew cask install virtualbox-extension-pack
$ brew cask install vagrant
$ brew install ansible
\end{minted}

\section{Werkvormen}%
\label{sec:werkvormen}

In dit opleidingsonderdeel wordt hoofdzakelijk gewerkt aan de hand van labo-op\-drach\-ten. De ondersteunende studiematerialen (zie Sectie~\ref{sec:leermateriaal}) en deze syllabus moeten je in staat stellen die met succes af te werken.

\textbf{Tijdens de reguliere lessen} geven de lectoren klassikaal instructie waarna studenten zelfstandig en op eigen tempo werken aan de opdrachten. Studenten hebben op regelmatige basis een voortgangsgesprek met de lector en kunnen dan uitleg vragen bij specifieke problemen.

Als de contactmomenten omwille van corona-maatregelen niet op de campus kunnen doorgaan, dan gebeuren zowel de klassikale instructie als individuele opvogingsgesprekken via Microsoft Teams (videoconferencing).

De lector zal elke wijziging in deze regeling via Chamilo aankondigen, maar je mag in principe uitgaan van volgende regeling:

\begin{itemize}
  \item \textbf{Code groen/geel}: contactmomenten gaan door op de campus
  \item \textbf{Code oranje/rood}: contactmomenten gaan door via Teams
\end{itemize}

Buiten de contactmomenten kunnen studenten vragen stellen via Teams (chat-kanaal van het Team voor dit vak). Studenten worden aangemoedigd om daar ook vragen van anderen te beantwoorden.

Studenten \textbf{afstandsleren} kunnen eveneens vragen stellen via Teams chat of tijdens de contactmomenten voor TILE\@. Als deze contactmomenten niet op de campus doorgaan, gebeurt dit via Teams (videoconferencing).

\section{Werk- en leeraanwijzingen}%
\label{sec:werk-en-leeraanwijzingen}

Het werken met labo-opdrachten vergt een zekere mate van zelfstandigheid van jou als student, maar dat is precies ook een attitude die verwacht wordt van een systeembeheerder. Je neemt dus zelf initiatief om de nodige kennis te vergaren en zoekt naar oplossingen voor de problemen die je ongetwijfeld zal tegenkomen. Help elkaar daarin: samenwerken en kennis delen wordt van harte aangemoedigd. De lector is uiteraard beschikbaar om je bij te staan als je vast komt te zitten en kan je tips geven of verwijzen naar geschikte aanvullende studiematerialen.

\textbf{Reguliere studenten} moeten minstens eens in de drie weken persoonlijk bij de lector langs komen om deelresultaten te tonen. Kom zeker langs als je ergens vast zit, zodat de lector je terug op weg kan helpen!

\section{Studiebegeleiding en planning}%
\label{sec:studiebegeleiding-en-planning}

\textbf{Reguliere studenten} stellen vragen over de cursus bij voorkeur tijdens de contactmomenten of via Teams chat.

\textbf{Studenten afstandsleren} kunnen eveneens vragen stellen tijdens de voorziene contactmomenten waarop de lector aanwezig is, via Teams chat of via e-mail. Gelieve het onderwerp van je bericht te laten voorafgaan door de vermelding ``[ELNX]''.

Als je het antwoord kent op gestelde vragen op Teams mag je ook zelf antwoorden. Alvast bedankt in dat geval!

Tabel~\ref{tab:weekplanning} geeft een overzicht van de weekplanning voor de reguliere studenten. Studenten afstandsleren kunnen dit ook gebruiken als een leidraad bij het bewaken van de voortgang van hun studie.

\begin{table}
  \centering
  \begin{tabular}{ll}
    \toprule
    \textbf{Lesweek} & \textbf{Onderwerp}                                             \\
    \midrule
    1  & Inleiding, uitleg opdrachten                                   \\
       & Les + demo basiscommando's EL8, VirtualBox netwerkconfiguratie \\
    2  & Finale keuze hoofdopdracht                                     \\
       & Les + demo Ansible                                             \\
    3  & Individuele opvolging                                          \\
    4  & Presentatie/demo troubleshooting                               \\
    5  & Eerste troubleshooting-opdracht                                \\
    6  & Individuele opvolging                                          \\
    7  & Les + Demo BIND, Samba                                         \\
    8  & Individuele opvolging                                          \\
    9  & Individuele opvolging                                          \\
    10 & Individuele opvolging                                          \\
    11 & Tweede troubleshooting-opdracht                                \\
    12 & Individuele opvolging                                          \\
    13 & Inhaalsessie (Indien ingericht door je lector)
  \end{tabular}
  \caption{Weekplanning over het semester. Merk op dat er nog verschuivingen kunnen gebeuren naargelang er lessen wegvallen door verlofdagen, enz.}%
  \label{tab:weekplanning}
\end{table}

Voor \textbf{reguliere studenten} is aanwezigheid op de troubleshooting-labo's verplicht, aangezien dit evaluatiemomenten zijn. Bij ziekte volg je de normale procedure voor het wettigen van je afwezigheid. Contacteer ook zo snel mogelijk je lector voor een inhaalopdracht.

Bij het begin van een troubleshooting-opdracht krijg je een opstelling voorgeschoteld (typisch een virtuele machine) met configuratiefouten. Het is jouw taak die zo snel mogelijk systematisch op te sporen en op te lossen \textbf{volgens de methode die je aangeleerd krijgt}. Over dit proces en je resultaten schrijf je een laboverslag dat je indient op Chamilo.

\textbf{Studenten afstandsleren} kunnen met de lector tijdens die week een moment afspreken waarop je tijd hebt om hier aan te werken. Op het afgesproken tijdstip krijg je een downloadlink naar de opgave en dien je vóór de afgesproken deadline (in principe 4 uur na ontvangst van de opgave) je verslag in op Chamilo.

\section{Evaluatie}%
\label{sec:evaluatie}

De evaluatie van dit opleidingsonderdeel gebeurt voledig via permanente evaluatie. Meer bepaald word je beoordeeld op de manier waarop je volgende taken hebt uitgevoerd:

\begin{itemize}
  \item Hoofdopdracht: opzetten infrastructuur met een configuration management system
  \item Taak rond actualiteit in het vakgebied
  \item Troubleshooting van netwerkservices
  \item Rapportering en documentatie
\end{itemize}

Afhankelijk van de corona-maatregelen die op dat moment van kracht zijn, worden de evaluatiemomenten georganiseerd als volgt:

\begin{itemize}
  \item \textbf{Code groen/geel}: evaluatiemomenten gaan door op de campus
  \item \textbf{Code oranje/rood}: evaluatiemomenten gaan door via Microsoft Teams
\end{itemize}

De exacte organisatie wordt via Chamilo aangekondigd.

Troubleshooting wordt geëvalueerd op basis van vaardigheidstests, de andere taken op basis van een portfolio dat je samenstelt tijdens de loop van het semester en dat je op een evaluatiemoment tijdens de examenperiode komt verdedigen. Dat portfolio bestaat concreet uit volgende elementen:

\begin{itemize}
  \item De broncode, door elke student bijgehouden in een toegewezen Git repository
  \item Gedetailleerde labo-verslagen, eveneens bijgehouden in Git of desgevallend ingediend via Chamilo
  \item Demonstratie van deelresultaten aan de lector tijdens het semester
  \item Demonstratie van het eindresultaat aan de lector tijdens de examenperiode
\end{itemize}

In Hoofdstuk~\ref{ch:opdrachten} wordt in meer detail uitgelegd wat precies verwacht wordt.

De toekenning van het examencijfer gebeurt op basis van ``rubrics''~\autocite{Andrade2000} die beschreven worden in de evaluatiekaart die gepubliceerd wordt op Chamilo.

In een tabel worden een aantal criteria opgesomd, waar je aan moet voldoen. Je kan ``voldoen'' op verschillende niveaus, meer bepaald ``bekwaam'', ``gevorderd'', ``deskundig'', of in het slechtste geval ``nog niet bekwaam''. Voor elk niveau wordt beschreven wat dat precies inhoudt.

Om te slagen voor dit vak moet je aantonen dat je voor \textbf{alle} technische criteria minstens ``bekwaam'' bent. Met andere woorden, als je voor slechts één criterium ``nog niet bekwaam'' bevonden wordt, kan je niet slagen, hoe goed je ook de andere onderdelen hebt afgewerkt. De niet-technische criteria (bv.\ rapportering) kunnen het examencijfer afhankelijk van het behaalde niveau nog positief of negatief beïnvloeden.

\textbf{Merk op} dat je zowel een werkend product moet opleveren (= broncode) als de verslagen indienen én demo's geven. Als één van de deliverables ontbreekt, wordt de opdracht beschouwd als niet gerealiseerd.

\textbf{Reguliere studenten} moeten \emph{minstens elke drie weken} deelresulaten opleveren en persoonlijk demonstreren aan de lector. Deze periodes worden beschouwd als tussentijdse deadlines waarvoor art. 5 van het FOER van kracht is. Niet tijdig deelresultaten demonstreren wordt in termen van dat artikel beschouwd als het niet respecteren van tussentijdse deadlines en wordt als dusdanig gesanctioneerd. Een student die meer dan twee maal tussentijdse deadlines niet respecteert, krijgt 0 voor die opdracht (``Nog niet bekwaam'') en kan bijgevolg ook niet slagen.

Ook als je weinig tot niets gerealiseerd hebt, kom je langs. Dat is immers een teken dat je ergens vast zit, en de lector kan je dan opnieuw op weg helpen.

\textbf{Studenten afstandsleren} kunnen op verschillende manieren deelresultaten opleveren:

\begin{itemize}
  \item aan de hand van een screencast (bv.\ publiceren via Youtube of Panopto, link doorsturen naar de lector),
  \item via videoconferencing (Microsoft Teams, na afspraak),
  \item tijdens de contactmomenten voor TILE-studenten, als de lector daar aanwezig is (en enkel na afspraak),
\end{itemize}

\subsection{Tweede examenkans}%
\label{subs:tweede-examenkans}

Wie niet slaagt, krijgt een tweede examenkans in de vorm van een individuele opdracht. De precieze opdracht hangt af van je evaluatie. Meer bepaald zal je moet kunnen aantonen dat je voor alle technische criteria minstens ``bekwaam'' bent. Deze opdracht wordt meteen vastgelegd op het finale evaluatiemoment.

Als je nog geen individuele opdracht gekregen hebt op het evaluatiemoment (bijvoorbeeld omdat je afwezig was), dan kom je naar de feedback. Studenten afstandsleren kunnen dit ook per e-mail regelen of een afspraak maken om dit te bespreken via Teams.

\textbf{Wie na afloop van de feedback nog geen afspraak gemaakt heeft voor een individuele opdracht, geeft daarmee te kennen niet te willen deelnemen aan de 2e examenkans.}